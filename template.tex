\RequirePackage{filecontents}
\begin{filecontents*}{\jobname.mst}
	delim_0 "\\IndexDotfill "
	delim_1 "\\IndexDotfill "
	headings_flag 1
	heading_prefix "  \\IndexHeading{"
		heading_suffix "}\n"
\end{filecontents*}


\documentclass[12pt,twosided,titlepage]{report}

\usepackage{fancyhdr}
\usepackage{graphicx}
\usepackage{psfrag}
\usepackage{cancel}
\usepackage{imakeidx}
\usepackage[hidelinks]{hyperref}
\usepackage{enumitem}
\usepackage{graphicx}
\usepackage{amssymb}
\usepackage{booktabs}
\usepackage{parskip}
\usepackage{flafter}
\usepackage{makecell}
\usepackage{tikz}
\usepackage{dsfont}
\usepackage{tcolorbox, xcolor}
\PassOptionsToPackage{svgnames}{xcolor}
\tcbuselibrary{theorems}
\usepackage{fancyhdr}
\tcbuselibrary{skins,breakable}
\usetikzlibrary{shadings,shadows}
\usepackage[Sonny]{fncychap}
%\usepackage{palatino} % this is the recommended font
% your other packages go here
\usepackage{amsmath,epsfig,latexsym,amssymb,amsthm, rotating}
\usepackage[ruled,linesnumbered]{algorithm2e}
\usepackage{calligra}

% Define some useful tools for these notes
% your latex commands go here

\usepackage[a4paper,width=150mm,top=25mm,bottom=25mm]{geometry}


\setlength{\parskip}{0.1in}
\setlength{\parindent}{0.3in}

\let\drop\smallsetminus
\let\oldemptyset\emptyset
\let\bar\overline
\let\emptyset\varnothing


% color

\definecolor{DarkGreen}{HTML}{305f72}
\definecolor{LightGreen}{HTML}{f1d1b5}
\definecolor{LimeGreen}{rgb}{0.2, 0.8, 0.2}
\definecolor{azure}{rgb}{0.8, 1.0, 1.0}
\definecolor{ocre}{HTML}{F16723}
\definecolor{DarkOrange}{HTML}{FF8C00}

\definecolor{lemmaback}{HTML}{fcf9ea}
\definecolor{lemmatop}{HTML}{badfdb}

\definecolor{corback}{HTML}{f1d4d4}
\definecolor{cortop}{HTML}{484c7f}

\definecolor{thmback}{HTML}{f0f5f9}
\definecolor{thmtop}{HTML}{52616b}

\definecolor{propback}{HTML}{eeeeee}
\definecolor{proptop}{HTML}{1089ff}

% http://ctan.org/pkg/listings
\usepackage{listings}
\lstset{
	numbers=left,
	tabsize=2,
	basicstyle=\ttfamily,
	mathescape,
	commentstyle     = \color{olive},	
}

% customized index
\newcommand*{\IndexDotfill}{%
	\nobreak\dotfill\ \nobreak
}
\renewcommand*{\indexspace}{%
	\par
	\vspace{25pt plus 6pt minus 4pt}%
}
\newcommand*{\IndexHeading}[1]{%
	\tikz\node[
	rounded corners=5pt,
	draw=ocre,
	fill=ocre!10,
	line width=1pt,
	inner sep=5pt,
	align=center,
	font=\sffamily\bfseries\large,
	minimum width=\linewidth-\pgflinewidth,
	] {#1};%
	\nopagebreak
	\par
	\vspace{2mm}%
}

% mdframed
\usepackage{calc}                                          
\usepackage[backgroundcolor=azure,
linecolor=red!60!black,
linewidth=2pt
]{mdframed}

\mdfsetup{frametitlealignment=\center}
\usetikzlibrary{decorations.markings}
\usepackage{float}
\restylefloat{table}


\setlength{\parskip}{0.15in}
\setlength{\parindent}{0in}


% customed chi
\makeatletter
\@ifdefinable\@latex@chi{\let\@latex@chi\chi}
\renewcommand*\chi{{\@latex@chi\smash[t]{\mathstrut}}} % want only bottom half of \mathstrut
\makeatletter

% marked text
\tcbset{textmarker/.style={%
		skin=enhancedmiddle,breakable,parbox=false,
		boxrule=0mm,leftrule=5mm,rightrule=5mm,boxsep=0mm,arc=0mm,outer arc=0mm,
		left=3mm,right=3mm,top=1mm,bottom=1mm,enlarge left by=-8mm,enlarge right by=-8mm,
		toptitle=1mm,bottomtitle=1mm,width=\the\dimexpr\linewidth+1.6cm\relax}}


%%%%%%%%%%%%%%%%%%%%% my blocks  %%%%%%%%%%%%%
\newcounter{mythmctr}



% defn block, don't need counter here
\newenvironment{defn}[1]{%
	\tcolorbox[beamer,%
	noparskip,breakable,
	colback=LightGreen,colframe=DarkGreen,%
	colbacklower=LimeGreen!75!LightGreen,%
	title={#1}] \index{#1} \label{defn:#1}
}%
{\endtcolorbox}



% theorem
\newenvironment{thm}[1][]{
	\refstepcounter{mythmctr}% increment the environment's counter
	\tcolorbox[beamer,%
	noparskip,breakable,
	colback=thmback,colframe=thmtop,%
	title={\bfseries Theorem \themythmctr{\ifthenelse{\equal{#1}{}}{}{: #1}}}]  \label{thm:\themythmctr}
}%
{\endtcolorbox}

% lemma, share theorem counter
\newenvironment{lemma}[1][]{
	\refstepcounter{mythmctr}% increment the environment's counter
	\tcolorbox[beamer,%
	noparskip,breakable,
	colback=lemmaback,colframe=lemmatop,%
	title={\bfseries Lemma \themythmctr{\ifthenelse{\equal{#1}{}}{}{: #1}}}]  \label{thm:\themythmctr}
}%
{\endtcolorbox}

% corrolary
\newenvironment{corr}[1][]{
	\refstepcounter{mythmctr}% increment the environment's counter
	\tcolorbox[beamer,%
	noparskip,breakable,
	colback=corback,colframe=cortop,%
	title={\bfseries Corrollary \themythmctr{\ifthenelse{\equal{#1}{}}{}{: #1}}}]  \label{thm:\themythmctr}
}%
{\endtcolorbox}


\newenvironment{prop}[1][]{
	\refstepcounter{mythmctr}% increment the environment's counter
	\tcolorbox[beamer,%
	noparskip,breakable,
	colback=propback,colframe=proptop,%
	title={\bfseries Proposition \themythmctr{\ifthenelse{\equal{#1}{}}{}{: #1}}}]  \label{thm:\themythmctr}
}%
{\endtcolorbox}

\numberwithin{mythmctr}{chapter}


% equation box: might need for amath courses
% note that you need equation environment in between
\newenvironment{eqbox}[1][]{
	\tcolorbox[textmarker,colback=DarkOrange!5!white,
	colframe=DarkOrange!75!yellow]
	\textit{#1}
}%
{\endtcolorbox}


\newenvironment{pf}{%
	{\calligra Proof:}
	\tcolorbox[blanker,breakable,left=5mm,
	before skip=10pt,after skip=10pt,
	borderline west={1mm}{0pt}{red}]
}%
{\qed \endtcolorbox}

\newenvironment{ex}{%
	{\calligra Example:}
	\tcolorbox[blanker,breakable,left=5mm,
	before skip=10pt,after skip=10pt,
	borderline west={1mm}{0pt}{blue}]
}%
{\endtcolorbox}

\newenvironment{remark}{%
	{\calligra Remark:}
	\tcolorbox[blanker,breakable,left=5mm,
	before skip=10pt,after skip=10pt,
	borderline west={1mm}{0pt}{pink}]
}%
{\endtcolorbox}


\newcommand{\defnref}[1]{\hyperref[defn:#1]{#1}}
\newcommand{\thmref}[1]{Theorem \ref{thm:#1}}
\newcommand{\lemmaref}[1]{Lemma \ref{thm:#1}}
\newcommand{\corref}[1]{Corollary \ref{thm:#1}}
\newcommand{\propref}[1]{Proposition \ref{thm:#1}}
% make index
\makeindex





% Title Page
\title{Introduction to Mathematics}
\author{Sibelius Peng}
\date{Fall 2019}


\begin{document}
\maketitle
{
	\hypersetup{linkbordercolor=white}
	\tableofcontents
}

\newpage
\pagestyle{headings}


\chapter{Schubert Calculus}
\begin{defn}{Schubert calculus}
In mathematics, Schubert calculus is a branch of algebraic geometry introduced in the nineteenth century by Hermann Schubert, in order to solve various counting problems of projective geometry.
\end{defn}

\begin{thm}[Pieri's formula]
Let $\sigma_\lambda$ be a special Schubert cycle and suppose $\sigma_\mu$ is any Schubert cycle with $\mu= \{\mu_1,\ldots,\mu_k\}$. Then we have 
$$ \sigma_\lambda\cdot \sigma_\mu = \sum \sigma_\nu$$
where the sum is over all $\nu$ such that $\mu_\ell \leqslant \nu_\ell \leqslant \mu_{\ell-1}$ and $\sum \nu_\ell -\mu_\ell = \lambda$.
\end{thm}

\begin{pf}
Simply invoke the duality theorem.
\end{pf}

\begin{remark}
\thmref{1.1} is an import theorem\footnote{theorem reference testing.}. Let's look at an example.
\end{remark}

\begin{ex}
Click \href{http://www.macs.hw.ac.uk/~simonm/schubertcalculusreview.pdf#page=14}{here} for details.
\end{ex}

Now let's look at a new theorem!

\begin{thm}
	For a fixed $k$, as $n\to\infty$, the following asymptotic holds:
	$$
	\log \operatorname{edge} ~ G(k,n)= kn \log\left(
	{\sqrt \pi \Gamma \left({k+1\over 2}\right) \over \Gamma \left({k\over 2}\right)}
	 \right)- \mathcal O (\log n)
	$$
\end{thm}

A general, guiding theme of our work is to compare the number of complex solutions with
the expected number of real solutions. With this regard, we obtain:

\begin{corr}
For fixed $k\ge 2$, we have $\operatorname{edge} G(k,n)=\deg G_{\mathbb C} (k,n)^{{1\over 2}\varepsilon_k+o(1)}$ for $n\to\infty$, where $\varepsilon_k$ is monotonically decreasing with $\lim_{k\to\infty}=1$. 
\end{corr}

\begin{remark}
This means that for large $n$, the expected degree of the real Grassmannian exceeds the square
root of the degree of the corresponding complex Grassmannian...
\end{remark}

\begin{lemma}
Let $K\subseteq \mathbb R^d$
be a convex body containing the origin. Then the radial function $_K$
and the support function $h_K$ of $K$ have the same maximum on $S^{d-1}$
. Moreover, a direction
$u\in S^{d-1}$
is maximizing for $r_K$ if and only u is maximizing for $h_K$.
\end{lemma}

\begin{pf}
Let $\overline{r}$ denote the maximum of $r_K$. Then $K\subseteq B(0, \overline r)$ and hence $h_K \le h_{B(0,\overline r)}$, which implies $\max h_K\le \max h_{B(0,\overline r)}= \overline r$. For the other direction let $u\in S^{d-1}$. Then ... The claim about the maximizing directions follows easily by tracing our argument.
\end{pf}

Now let's look at a proposition.

\begin{prop}
The set $\operatorname{Sing}(S)$ is semialgebraic and $\dim \operatorname{Sing}(S) < \dim S$. In particular, $S\ne \operatorname{Sing}(S)$.
\end{prop}
\begin{remark}
In the sequel it will be convenient to say that generic points of a semialgebraic set $S$ satisfy
a certain property if this property is satisfied by all points except in a semialgebraic subset of
positive codimension in $S$. (For example, in view of the previous proposition, generic points of
a semialgebraic set $S$ are regular points.)
\end{remark}


\chapter{General Relativity}
Now let's apply our math knowledge to physics!

Moving from math to physics involves the introduction of dynamical equations which relate
matter and energy to the curvature of spacetime. In GR, the “equation of motion” for the
metric is the famous \textbf{Einstein equation}:

\begin{eqbox}[Einstein equation]
\begin{equation}
R_{\mu\nu}-{1\over 2} R g_{\mu \nu} = 8\pi GT_{\mu\nu}
\end{equation}
\end{eqbox}

\begin{defn}{Kruskal coordinates}
We can
demonstrate this by making a transformation to what are known as \textbf{Kruskal coordinates},
defined by
\begin{equation}
\begin{split}
u &= \left( {r\over 2Gm}-1\right)^{1/2}e^{r/4Gm}\cosh(t/4Gm)\\
v &= \left( {r\over 2Gm}-1\right)^{1/2}e^{r/4Gm}\sinh(t/4Gm)
\end{split}
\end{equation}
\end{defn}

\begin{remark}
The useful thing about the Schwarzschild solution is that it describes both mundane
things like the solar system, and more exotic objects like black holes. To get a feel for it,
let’s look at how particles move in a Schwarzschild geometry. It turns out that we can cast
the problem of a particle moving in the plane  $\theta=\pi\over 2$ as a one-dimensional problem for the radial coordinate $r=r(\tau)$.
\end{remark}

\section{Cosmology}
Just as we were able to make great strides with the Schwarzschild metric on the assumption
of sperical symmetry, we can make similar progress in cosmology by assuming that the
Universe is homogeneous and isotropic. That is to say, we assume the existence of a “rest
frame for the Universe,” which defines a universal time coordinate, and singles out three-dimensional surfaces perpendicular to this time coordinate. (In the real Universe, this rest
frame is the one in which galaxies are at rest and the microwave background is isotropic.)
“Homogeneous” means that the curvature of any two points at a given time t is the same.
“Isotropic” is trickier, but basically means that the universe looks the same in all directions.
Thus, the surface of a cylinder is homogeneous (every point is the same) but not isotropic
(looking along the long axis of the cylinder is a preferred direction); a cone is isotropic around
its vertex, but not homogeneous.


\printindex
\end{document}
